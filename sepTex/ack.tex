% sepTex/ack.tex  -- UTF-8, no \documentclass or \begin{document}
\begin{acknowledgement}

站在硕士生涯的另一端,我该如何描述自己的故事呢?这三年来的,无数次心跳、兴奋与挣扎、倦怠,终将被我平淡的描述。我的初心是——喜欢电子信息相关的研究;觉得读硕士很体面;痴迷于上海的霓虹... ... 硕士生涯给了我更多——我学会了面对压力冷静地分析问题;学会了在繁杂的任务下安排时间;学会了一个人设计、执行实验;学会了包装自己与学术表达;树立了想在科研上更进一步的决心;将自己塑造成一个更加完善的人。这一路有贵人相助,有父母支持,有爱人相伴,有朋友鼓励,这些力量让我有可能、有能力在上海科技大学这个严格的环境下,完成自己的硕士学习。在此我将引入几个故事,表达我诚挚的感谢。


首先感谢我的恩师梁俊睿教授。记得向梁老师吐露自己想继续在学术上的求索的那晚,我们聊到了凌晨。讲到苏黎世的偏僻城堡,没有智能手机时代的GPS导航,伯克利周围的森林... 我收获了很多,冶学经历、生活感触。对于研究思路与研究方法上的点拨;对于申请博士的支持,给与的帮助。


其次感谢我的父母,




% you can use \emoji if compiling with LuaHBTeX
I love battery-free IoT \emoji{hot-pepper} \emoji{pear} \emoji{keycap-ten}

\hfill Your text to align right.

\hfill 贰零贰伍年,十贰月。

\end{acknowledgement}
